\documentclass[a4paper,12pt]{article}
\usepackage[slovene]{babel}
\usepackage[utf8]{inputenc}
\usepackage[T1]{fontenc}
\usepackage{amsmath, amssymb}
\usepackage{amsthm}
\usepackage{graphicx, comment, caption, float}

\setlength{\belowcaptionskip}{5mm}
\setlength{\parindent}{0pt}
\setlength{\parskip}{8pt}

\newcommand{\program}{Finančna matematika} % ime studijskega programa: Matematika/Finančna matematika
\newcommand{\imeavtorja}{Nena Šefman Hodnik, Neo Mistral} % ime avtorja
\newcommand{\imementorja}{doc. dr. Janoš Vidali} % akademski naziv in ime mentorja
\newcommand{\imesomentorja}{prof. dr. Riste Škrekovski}
\newcommand{\naslovdela}{Sigma popolna iregularnost za grafe brez\\ciklov dolžine 3}
\newcommand{\letnica}{2025} %letnica 

\begin{document}

%naslovnica

\thispagestyle{empty}
\noindent{\large
UNIVERZA V LJUBLJANI\\[1mm]
FAKULTETA ZA MATEMATIKO IN FIZIKO\\[5mm]
\program\ }
\vfill

\begin{center}{\large
\imeavtorja\\[2mm]
{\bf \naslovdela}\\[10mm]
Skupinski projekt\\[2mm]
Poročilo\\[1cm]
Mentorja: \imementorja, \\ \imesomentorja\\[2mm]}
\end{center}
\vfill

\noindent{\large
Ljubljana, januar \letnica}
\pagebreak

%vsebina

\section{Uvod}
V teoriji grafov mere iregularnosti količinsko opredeljujejo, do kakšne mere so grafi iregularni. Tako vrnejo vrednost $0$ za 
regularne grafe, večja vrednost pa pomeni, da je graf bolj iregularen. Takih mer je ogromno, v tem projektu pa se bova osredotočila 
na sigma popolno iregularnost (v nadaljevanju STI), ki je definirana na naslednji način:

$$ \sigma_t(G) = \sum_{{\{u, v\}}\subset V(G)} (d_G(u) - d_G(v))^2. $$

Pri tem $d_G(v)$ označuje stopnjo vozlišča $v$ v grafu $G$, $V(G)$ pa je množica vseh vozlišč grafa. Sigma popolna iregularnost torej meri 
vsoto kvadratov razlik med stopnjami vseh parov vozlišč v grafu.

\subsection{Opis problema}
Poiskati želiva grafe reda $n$ brez ciklov dolžine 3, ki imajo največjo možno sigma popolno iregularnost. Najprej bova za grafe nižjega reda
sistematično generirala grafe brez ciklov dolžine 3 in poiskala tiste, ki imajo največjo sigma popolno iregularnost. Rezultate bova posplošila 
za splošen graf reda $n$ in formulirala hipotezo o optimalni strukturi grafov, ki maksimizira sigma popolno iregularnost. Postavljeno hipotezo 
bova testirala tako, da bova kandidate za optimalne grafe malo spremenila in nato za njih izračunala sigma popolno iregularnost. Če bo najina 
hipoteza pravilna, bi morala vedno dobiti manjšo vrednost. Za testiranje bova uporabila naslednje štiri različne metahevristične algoritme:
simulated annealing (v nadaljevanju SA), tabu search, variable neighborhood search (v nadaljevanju VNS) in mešanico algoritmov SA in VNS.

\newpage

\section{Postavljanje hipoteze}
Generiranja grafov sva se lotila na tri načine. Najprej sva sistematično generirala vse možne grafe z $n$ vozlišči,
preverila, da ne vsebujejo ciklov dolžine 3 ter izločila izomorfne grafe. Drugi pristop temelji na knjižnici NumPy, kjer so grafi 
predstavljeni z matrikami sosednosti. Navsezadnje sva uporabila SageMath in funkcijo $\texttt{nauty\_geng}$, ki učinkovito generira
vse grafe brez ciklov dolžine 3. Kode za to lahko najdete na repozitoriju v datotekah $\texttt{small\_graphs.ipynb}$ in 
$\texttt{small\_graphs\_sage.ipynb}$.

Po generiranju sva izračunala STI za majhne grafe (do 13 vozlišč), poiskala tiste, pri katerih je bila vrednost najvišja ter jih izrisala. 
Na podlagi dobljenih rezultatov sva oblikovala hipotezo, da so grafi z največjo STI zvezdasti grafi.

\subsection{Zvezdasti grafi}
Opazko glede oblike sva želela preveriti v datoteki $\texttt{star\_graphs.ipynb}$. S tem  namenom sva s funkcijo $\texttt{generate\_star\_graph}$
najprej generirala zvezdaste grafe z različnim številom osrednjih vozlišč. Nato sva med temi grafi poiskala tiste, ki imajo največjo vrednost STI 
ter na ta način dobila optimalno število osrednjih vozlišč. 

Na spodnjih slikah so prikazani optimalni zvezdasti grafi stopenj 10, 14 in 17, ki imajo največjo možno STI.

\begin{figure}[h]
      \centering
      \begin{minipage}[b]{0.32\textwidth}
          \centering
          \includegraphics[width=\textwidth]{graf_eno_centralno.png}
      \end{minipage}
      \hfill
      \begin{minipage}[b]{0.32\textwidth}
          \centering
          \includegraphics[width=\textwidth]{graf_dve_centralni.png}
      \end{minipage}
      \hfill
      \begin{minipage}[b]{0.32\textwidth}
          \centering
          \includegraphics[width=\textwidth]{graf_tri_centralna.png}
      \end{minipage}
  \end{figure}  

Zanimalo naju je tudi, kako narašča število centralnih vozlišč glede na stopnjo grafa in pa kakšno je razmerje med centralnimi 
in robnimi vozlišči. Opazila sva, da se število centralnih vozlišč enakomerno (skoraj linearno) povečuje in ima specifično 
stopničasto strukturo, skoki pa se pojavljajo na približno enakih razmakih. To si lahko razlagamo na način, da dodajanje novih 
vozlišč v graf najprej povečuje število robnih vozlišč, ko dosežemo določen prag, pa se doda še eno centralno vozlišče. 
Za boljšo vizualizacijo sva narisala tudi dva grafa.

\begin{figure}[h!]
      \includegraphics[width=\textwidth]{graf_st_centralnih.png}
      \includegraphics[width=\textwidth]{graf_razmerje_med_centralnimi_in_robnimi.png}
\end{figure}


\section{Testiranje hipotez}
\subsection{Metahevristični algoritmi}
\subsubsection{Simulated annealing}
Osnovna ideja algoritma Simulated annealing je, da v določenih primerih dovoljuje poslabšanja trenutne rešitve, kar omogoča pobeg
iz lokalnih minimumov. Algoritem se začne z generiranjem začetne rešitve (v najinem primeru sva vzela kar zvezdaste grafe z 
največjo možno STI), določimo pa tudi začetno temperaturo. Nato v vsaki iteraciji izberemo novo rešitev in jo sprejmemo glede na 
njeno kakovost v primerjavi s trenutno rešitvijo. Če je nova rešitev, boljša jo sprejmemo, sicer pa jo sprejmemo z določeno 
verjetnostjo (določeno z Boltzmanovo porazdelitvijo).Sčasoma se temperatura zmanjšuje, kar pomeni, da je na začetku verjetnost 
sprejemanja slabših rešitev visoka, nato pa čedalje nižja. Algoritem je enostaven za implementacijo ter zelo prilagodljiv različnim 
optimizacijskim problemom, prednost pa je tudi, da omogoča pobeg iz lokalnih minimumov. Vendar pa je uspešnost močno odvisna od 
začetne temperature in hitrosti ohlajanja, če temperatura upada prehitro ali prepočasi naši rezultati lahko niso točni.
Koda za opisani algoritem je shranjena v datoteki $\texttt{simulated\_annealing.ipynb}$.

\subsubsection{Tabu search}



\subsubsection{Variable neighborhood search}
\subsubsection{Mešani algoritem - SA in VNS}
\subsection{Ugotovitve}

\section{Zaključek}

\section{Viri}

\end{document}